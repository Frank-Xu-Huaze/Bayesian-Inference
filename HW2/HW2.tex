\documentclass[]{article}
\usepackage{lmodern}
\usepackage{amssymb,amsmath}
\usepackage{ifxetex,ifluatex}
\usepackage{fixltx2e} % provides \textsubscript
\ifnum 0\ifxetex 1\fi\ifluatex 1\fi=0 % if pdftex
  \usepackage[T1]{fontenc}
  \usepackage[utf8]{inputenc}
\else % if luatex or xelatex
  \ifxetex
    \usepackage{mathspec}
  \else
    \usepackage{fontspec}
  \fi
  \defaultfontfeatures{Ligatures=TeX,Scale=MatchLowercase}
\fi
% use upquote if available, for straight quotes in verbatim environments
\IfFileExists{upquote.sty}{\usepackage{upquote}}{}
% use microtype if available
\IfFileExists{microtype.sty}{%
\usepackage{microtype}
\UseMicrotypeSet[protrusion]{basicmath} % disable protrusion for tt fonts
}{}
\usepackage[margin=1in]{geometry}
\usepackage{hyperref}
\hypersetup{unicode=true,
            pdftitle={HW2},
            pdfauthor={Frank},
            pdfborder={0 0 0},
            breaklinks=true}
\urlstyle{same}  % don't use monospace font for urls
\usepackage{color}
\usepackage{fancyvrb}
\newcommand{\VerbBar}{|}
\newcommand{\VERB}{\Verb[commandchars=\\\{\}]}
\DefineVerbatimEnvironment{Highlighting}{Verbatim}{commandchars=\\\{\}}
% Add ',fontsize=\small' for more characters per line
\usepackage{framed}
\definecolor{shadecolor}{RGB}{248,248,248}
\newenvironment{Shaded}{\begin{snugshade}}{\end{snugshade}}
\newcommand{\KeywordTok}[1]{\textcolor[rgb]{0.13,0.29,0.53}{\textbf{#1}}}
\newcommand{\DataTypeTok}[1]{\textcolor[rgb]{0.13,0.29,0.53}{#1}}
\newcommand{\DecValTok}[1]{\textcolor[rgb]{0.00,0.00,0.81}{#1}}
\newcommand{\BaseNTok}[1]{\textcolor[rgb]{0.00,0.00,0.81}{#1}}
\newcommand{\FloatTok}[1]{\textcolor[rgb]{0.00,0.00,0.81}{#1}}
\newcommand{\ConstantTok}[1]{\textcolor[rgb]{0.00,0.00,0.00}{#1}}
\newcommand{\CharTok}[1]{\textcolor[rgb]{0.31,0.60,0.02}{#1}}
\newcommand{\SpecialCharTok}[1]{\textcolor[rgb]{0.00,0.00,0.00}{#1}}
\newcommand{\StringTok}[1]{\textcolor[rgb]{0.31,0.60,0.02}{#1}}
\newcommand{\VerbatimStringTok}[1]{\textcolor[rgb]{0.31,0.60,0.02}{#1}}
\newcommand{\SpecialStringTok}[1]{\textcolor[rgb]{0.31,0.60,0.02}{#1}}
\newcommand{\ImportTok}[1]{#1}
\newcommand{\CommentTok}[1]{\textcolor[rgb]{0.56,0.35,0.01}{\textit{#1}}}
\newcommand{\DocumentationTok}[1]{\textcolor[rgb]{0.56,0.35,0.01}{\textbf{\textit{#1}}}}
\newcommand{\AnnotationTok}[1]{\textcolor[rgb]{0.56,0.35,0.01}{\textbf{\textit{#1}}}}
\newcommand{\CommentVarTok}[1]{\textcolor[rgb]{0.56,0.35,0.01}{\textbf{\textit{#1}}}}
\newcommand{\OtherTok}[1]{\textcolor[rgb]{0.56,0.35,0.01}{#1}}
\newcommand{\FunctionTok}[1]{\textcolor[rgb]{0.00,0.00,0.00}{#1}}
\newcommand{\VariableTok}[1]{\textcolor[rgb]{0.00,0.00,0.00}{#1}}
\newcommand{\ControlFlowTok}[1]{\textcolor[rgb]{0.13,0.29,0.53}{\textbf{#1}}}
\newcommand{\OperatorTok}[1]{\textcolor[rgb]{0.81,0.36,0.00}{\textbf{#1}}}
\newcommand{\BuiltInTok}[1]{#1}
\newcommand{\ExtensionTok}[1]{#1}
\newcommand{\PreprocessorTok}[1]{\textcolor[rgb]{0.56,0.35,0.01}{\textit{#1}}}
\newcommand{\AttributeTok}[1]{\textcolor[rgb]{0.77,0.63,0.00}{#1}}
\newcommand{\RegionMarkerTok}[1]{#1}
\newcommand{\InformationTok}[1]{\textcolor[rgb]{0.56,0.35,0.01}{\textbf{\textit{#1}}}}
\newcommand{\WarningTok}[1]{\textcolor[rgb]{0.56,0.35,0.01}{\textbf{\textit{#1}}}}
\newcommand{\AlertTok}[1]{\textcolor[rgb]{0.94,0.16,0.16}{#1}}
\newcommand{\ErrorTok}[1]{\textcolor[rgb]{0.64,0.00,0.00}{\textbf{#1}}}
\newcommand{\NormalTok}[1]{#1}
\usepackage{graphicx,grffile}
\makeatletter
\def\maxwidth{\ifdim\Gin@nat@width>\linewidth\linewidth\else\Gin@nat@width\fi}
\def\maxheight{\ifdim\Gin@nat@height>\textheight\textheight\else\Gin@nat@height\fi}
\makeatother
% Scale images if necessary, so that they will not overflow the page
% margins by default, and it is still possible to overwrite the defaults
% using explicit options in \includegraphics[width, height, ...]{}
\setkeys{Gin}{width=\maxwidth,height=\maxheight,keepaspectratio}
\IfFileExists{parskip.sty}{%
\usepackage{parskip}
}{% else
\setlength{\parindent}{0pt}
\setlength{\parskip}{6pt plus 2pt minus 1pt}
}
\setlength{\emergencystretch}{3em}  % prevent overfull lines
\providecommand{\tightlist}{%
  \setlength{\itemsep}{0pt}\setlength{\parskip}{0pt}}
\setcounter{secnumdepth}{0}
% Redefines (sub)paragraphs to behave more like sections
\ifx\paragraph\undefined\else
\let\oldparagraph\paragraph
\renewcommand{\paragraph}[1]{\oldparagraph{#1}\mbox{}}
\fi
\ifx\subparagraph\undefined\else
\let\oldsubparagraph\subparagraph
\renewcommand{\subparagraph}[1]{\oldsubparagraph{#1}\mbox{}}
\fi

%%% Use protect on footnotes to avoid problems with footnotes in titles
\let\rmarkdownfootnote\footnote%
\def\footnote{\protect\rmarkdownfootnote}

%%% Change title format to be more compact
\usepackage{titling}

% Create subtitle command for use in maketitle
\newcommand{\subtitle}[1]{
  \posttitle{
    \begin{center}\large#1\end{center}
    }
}

\setlength{\droptitle}{-2em}

  \title{HW2}
    \pretitle{\vspace{\droptitle}\centering\huge}
  \posttitle{\par}
  \subtitle{STA-360/602, Spring 2018}
  \author{Frank}
    \preauthor{\centering\large\emph}
  \postauthor{\par}
      \predate{\centering\large\emph}
  \postdate{\par}
    \date{1/16/2019}


\begin{document}
\maketitle

\paragraph{1. Lab component}\label{lab-component}

\paragraph{\texorpdfstring{(a) Task 3. Write a function that takes as
its inputs that data you simulated (or any data of the same type) and a
sequence of \(\theta\) values of length 1000 and produces Likelihood
values based on the Binomial Likelihood. Plot your sequence and its
corresponding Likelihood
function.}{(a) Task 3. Write a function that takes as its inputs that data you simulated (or any data of the same type) and a sequence of \textbackslash{}theta values of length 1000 and produces Likelihood values based on the Binomial Likelihood. Plot your sequence and its corresponding Likelihood function.}}\label{a-task-3.-write-a-function-that-takes-as-its-inputs-that-data-you-simulated-or-any-data-of-the-same-type-and-a-sequence-of-theta-values-of-length-1000-and-produces-likelihood-values-based-on-the-binomial-likelihood.-plot-your-sequence-and-its-corresponding-likelihood-function.}

\begin{Shaded}
\begin{Highlighting}[]
\CommentTok{# Generating simulated data}
\KeywordTok{set.seed}\NormalTok{(}\DecValTok{123}\NormalTok{)}
\NormalTok{obs.data <-}\StringTok{ }\KeywordTok{rbinom}\NormalTok{(}\DataTypeTok{n =} \DecValTok{100}\NormalTok{, }\DataTypeTok{size =} \DecValTok{1}\NormalTok{, }\DataTypeTok{prob =} \FloatTok{0.01}\NormalTok{)}
\KeywordTok{head}\NormalTok{(obs.data)}
\end{Highlighting}
\end{Shaded}

\begin{verbatim}
## [1] 0 0 0 0 0 0
\end{verbatim}

\begin{Shaded}
\begin{Highlighting}[]
\CommentTok{# Create and plot likelihood function}
\NormalTok{n =}\StringTok{ }\KeywordTok{length}\NormalTok{(obs.data)}
\NormalTok{x =}\StringTok{ }\KeywordTok{sum}\NormalTok{(obs.data)}
\NormalTok{th =}\StringTok{ }\KeywordTok{seq}\NormalTok{(}\DecValTok{0}\NormalTok{, }\DecValTok{1}\NormalTok{, }\DataTypeTok{length =} \DecValTok{1000}\NormalTok{)}

\NormalTok{like =}\StringTok{ }\KeywordTok{dbeta}\NormalTok{(th, x}\OperatorTok{+}\DecValTok{1}\NormalTok{, n}\OperatorTok{-}\NormalTok{x}\OperatorTok{+}\DecValTok{1}\NormalTok{)}
\KeywordTok{ggplot}\NormalTok{()}\OperatorTok{+}\KeywordTok{geom_line}\NormalTok{(}\KeywordTok{aes}\NormalTok{(}\DataTypeTok{x =}\NormalTok{ th, }\DataTypeTok{y =}\NormalTok{ like))}\OperatorTok{+}\KeywordTok{labs}\NormalTok{(}\DataTypeTok{x =} \KeywordTok{expression}\NormalTok{(theta), }\DataTypeTok{y =} \StringTok{"Density"}\NormalTok{)}
\end{Highlighting}
\end{Shaded}

\includegraphics{HW2_files/figure-latex/unnamed-chunk-1-1.pdf}

\paragraph{\texorpdfstring{(b) Task 4. Write a function that takes as
its inputs prior parameters a and b for the Beta-Bernoulli model and the
observed data, and produces the posterior parameters you need for the
model. Generate the posterior parameters for a non-informative prior
i.e. \((a,b) = (1,1)\) and for an informative case
\((a,b) = (3,1)\)}{(b) Task 4. Write a function that takes as its inputs prior parameters a and b for the Beta-Bernoulli model and the observed data, and produces the posterior parameters you need for the model. Generate the posterior parameters for a non-informative prior i.e. (a,b) = (1,1) and for an informative case (a,b) = (3,1)}}\label{b-task-4.-write-a-function-that-takes-as-its-inputs-prior-parameters-a-and-b-for-the-beta-bernoulli-model-and-the-observed-data-and-produces-the-posterior-parameters-you-need-for-the-model.-generate-the-posterior-parameters-for-a-non-informative-prior-i.e.-ab-11-and-for-an-informative-case-ab-31}

\begin{Shaded}
\begin{Highlighting}[]
\NormalTok{a1 =}\StringTok{ }\DecValTok{1}
\NormalTok{b1 =}\StringTok{ }\DecValTok{1}
\NormalTok{a2 =}\StringTok{ }\DecValTok{3}
\NormalTok{b2 =}\StringTok{ }\DecValTok{1}

\NormalTok{prior1 =}\StringTok{ }\KeywordTok{dbeta}\NormalTok{(th, a1, b1)}
\NormalTok{post1 =}\StringTok{ }\KeywordTok{dbeta}\NormalTok{(th, a1}\OperatorTok{+}\NormalTok{x, b1}\OperatorTok{+}\NormalTok{n}\OperatorTok{-}\NormalTok{x)}

\NormalTok{prior2 =}\StringTok{ }\KeywordTok{dbeta}\NormalTok{(th, a2, b2)}
\NormalTok{post2 =}\StringTok{ }\KeywordTok{dbeta}\NormalTok{(th, a2}\OperatorTok{+}\NormalTok{x, b2}\OperatorTok{+}\NormalTok{n}\OperatorTok{-}\NormalTok{x)}
\end{Highlighting}
\end{Shaded}

\paragraph{(b) Task 5. Create two plots, one for the informative and one
for the non-informative case to show the posterior distribution and
superimpose the prior distribu- tions on each along with the likelihood.
What do you see? Remember to turn the y-axis ticks off since
superimposing may make the scale
non-sense.}\label{b-task-5.-create-two-plots-one-for-the-informative-and-one-for-the-non-informative-case-to-show-the-posterior-distribution-and-superimpose-the-prior-distribu--tions-on-each-along-with-the-likelihood.-what-do-you-see-remember-to-turn-the-y-axis-ticks-off-since-superimposing-may-make-the-scale-non-sense.}

\begin{Shaded}
\begin{Highlighting}[]
\CommentTok{# Plot for non-informative case}
\KeywordTok{ggplot}\NormalTok{() }\OperatorTok{+}\StringTok{ }
\StringTok{  }\KeywordTok{geom_line}\NormalTok{(}\KeywordTok{aes}\NormalTok{(}\DataTypeTok{x =}\NormalTok{ th, }\DataTypeTok{y =}\NormalTok{ post1, }\DataTypeTok{linetype =} \StringTok{'posterior'}\NormalTok{))}\OperatorTok{+}
\StringTok{  }\KeywordTok{geom_line}\NormalTok{(}\KeywordTok{aes}\NormalTok{(}\DataTypeTok{x =}\NormalTok{ th, }\DataTypeTok{y =}\NormalTok{ prior1, }\DataTypeTok{linetype =} \StringTok{'prior'}\NormalTok{))}\OperatorTok{+}
\StringTok{  }\KeywordTok{geom_line}\NormalTok{(}\KeywordTok{aes}\NormalTok{(}\DataTypeTok{x =}\NormalTok{ th, }\DataTypeTok{y =}\NormalTok{ like, }\DataTypeTok{linetype =} \StringTok{'likelihood'}\NormalTok{))}\OperatorTok{+}
\StringTok{  }\KeywordTok{labs}\NormalTok{(}\DataTypeTok{x =} \KeywordTok{expression}\NormalTok{(theta), }\DataTypeTok{y =} \StringTok{"Density"}\NormalTok{, }\DataTypeTok{title =} \StringTok{"Prior, likelihood and posterior when given a non-informative prior"}\NormalTok{) }\OperatorTok{+}\StringTok{ }\KeywordTok{theme_bw}\NormalTok{()}
\end{Highlighting}
\end{Shaded}

\includegraphics{HW2_files/figure-latex/unnamed-chunk-3-1.pdf}

\begin{Shaded}
\begin{Highlighting}[]
\CommentTok{# Plot for informative case}
\KeywordTok{ggplot}\NormalTok{() }\OperatorTok{+}\StringTok{ }
\StringTok{  }\KeywordTok{geom_line}\NormalTok{(}\KeywordTok{aes}\NormalTok{(}\DataTypeTok{x =}\NormalTok{ th, }\DataTypeTok{y =}\NormalTok{ post2, }\DataTypeTok{linetype =} \StringTok{'posterior'}\NormalTok{))}\OperatorTok{+}
\StringTok{  }\KeywordTok{geom_line}\NormalTok{(}\KeywordTok{aes}\NormalTok{(}\DataTypeTok{x =}\NormalTok{ th, }\DataTypeTok{y =}\NormalTok{ prior2, }\DataTypeTok{linetype =} \StringTok{'prior'}\NormalTok{))}\OperatorTok{+}
\StringTok{  }\KeywordTok{geom_line}\NormalTok{(}\KeywordTok{aes}\NormalTok{(}\DataTypeTok{x =}\NormalTok{ th, }\DataTypeTok{y =}\NormalTok{ like, }\DataTypeTok{linetype =} \StringTok{'likelihood'}\NormalTok{))}\OperatorTok{+}
\StringTok{  }\KeywordTok{labs}\NormalTok{(}\DataTypeTok{x =} \KeywordTok{expression}\NormalTok{(theta), }\DataTypeTok{y =} \StringTok{"Density"}\NormalTok{, }\DataTypeTok{title =} \StringTok{"Prior, likelihood and posterior when given an informative prior"}\NormalTok{) }\OperatorTok{+}\StringTok{ }\KeywordTok{theme_bw}\NormalTok{()}
\end{Highlighting}
\end{Shaded}

\includegraphics{HW2_files/figure-latex/unnamed-chunk-3-2.pdf}

\paragraph{\texorpdfstring{2. The \textbf{\emph{Exponential-Gamma
Model}} We write \(X\sim Exp(\theta)\) to indicate that \(X\) has the
Exponential distribution, that is, its
p.d.f.~is}{2. The Exponential-Gamma Model We write X\textbackslash{}sim Exp(\textbackslash{}theta) to indicate that X has the Exponential distribution, that is, its p.d.f.~is}}\label{the-exponential-gamma-model-we-write-xsim-exptheta-to-indicate-that-x-has-the-exponential-distribution-that-is-its-p.d.f.is}

\[ p(x|\theta) = Exp(x|\theta) = \theta\exp(-\theta x)1(x>0). \]
\#\#\#\# The Exponential distribution has some special properties that
make it a good model for certain applications. It has been used to model
the time between events (such as neuron spikes, website hits, neutrinos
captured in a detector), extreme values such as maximum daily rainfall
over a period of one year, or the amount of time until a product fails
(lightbulbs are a standard example).

\paragraph{\texorpdfstring{Suppose you have data \(x_1,\dotsc,x_n\)
which you are modeling as i.i.d.~observations from an Exponential
distribution, and suppose that your prior is \(\theta\sim Gamma(a,b)\),
that
is,}{Suppose you have data x\_1,\textbackslash{}dotsc,x\_n which you are modeling as i.i.d.~observations from an Exponential distribution, and suppose that your prior is \textbackslash{}theta\textbackslash{}sim Gamma(a,b), that is,}}\label{suppose-you-have-data-x_1dotscx_n-which-you-are-modeling-as-i.i.d.observations-from-an-exponential-distribution-and-suppose-that-your-prior-is-thetasim-gammaab-that-is}

\[ p(\theta) = Gamma(\theta|a,b) = \frac{b^a}{\Gamma(a)}\theta^{a-1}\exp(-b\theta)1(\theta>0). \]

\paragraph{\texorpdfstring{(a) Derive the formula for the posterior
density, \(p(\theta|x_{1:n})\). Give the form of the posterior in terms
of one of the most common distributions (Bernoulli, Beta, Exponential,
or
Gamma).}{(a) Derive the formula for the posterior density, p(\textbackslash{}theta\textbar{}x\_\{1:n\}). Give the form of the posterior in terms of one of the most common distributions (Bernoulli, Beta, Exponential, or Gamma).}}\label{a-derive-the-formula-for-the-posterior-density-pthetax_1n.-give-the-form-of-the-posterior-in-terms-of-one-of-the-most-common-distributions-bernoulli-beta-exponential-or-gamma.}

\[\begin{align}p(\theta|x_{1:n}) & \propto p(x_{1:n}|\theta)p(\theta)\\
& \propto \prod_n[\theta\exp(-\theta x)]\frac{b^a}{\Gamma(a)}\theta^{a-1}\exp(-b\theta)\\
& \propto \theta^n\exp(-n\bar{x}\theta)\theta^{a-1}\exp(-b\theta)\\
& \propto \theta^{n+a-1}\exp[-(n\bar{x}+b)\theta]\\
& \propto Gamma(\theta|n+a,n\bar{x}+b)
\end{align}\]

\paragraph{\texorpdfstring{(b) Why is the posterior distribution a
\textbf{\emph{proper}} density or probability distribution
function?}{(b) Why is the posterior distribution a proper density or probability distribution function?}}\label{b-why-is-the-posterior-distribution-a-proper-density-or-probability-distribution-function}

Because the prior \(p(\theta) = Gamma(\theta|a,b)\) is a conjugate prior
for the given likelihood function. That means the posterior is in the
same probability distribution family as the prior probability
distribution \(p(\theta)\) is.

\paragraph{\texorpdfstring{(c) Now, suppose you are measuring the number
of seconds between lightning strikes during a storm, your prior is
\(Gamma(0.1,1.0)\), and your data
is}{(c) Now, suppose you are measuring the number of seconds between lightning strikes during a storm, your prior is Gamma(0.1,1.0), and your data is}}\label{c-now-suppose-you-are-measuring-the-number-of-seconds-between-lightning-strikes-during-a-storm-your-prior-is-gamma0.11.0-and-your-data-is}

\[(x_1,\dotsc,x_8) = (20.9, 69.7, 3.6, 21.8, 21.4, 0.4, 6.7, 10.0).\]
Plot the prior and posterior p.d.f.s. (Be sure to make your plots on a
scale that allows you to clearly see the important features.)

\begin{Shaded}
\begin{Highlighting}[]
\NormalTok{X =}\StringTok{ }\KeywordTok{c}\NormalTok{(}\FloatTok{20.9}\NormalTok{, }\FloatTok{69.7}\NormalTok{, }\FloatTok{3.6}\NormalTok{, }\FloatTok{21.8}\NormalTok{, }\FloatTok{21.4}\NormalTok{, }\FloatTok{0.4}\NormalTok{, }\FloatTok{6.7}\NormalTok{, }\FloatTok{10.0}\NormalTok{)}
\NormalTok{th =}\StringTok{ }\KeywordTok{seq}\NormalTok{(}\DecValTok{0}\NormalTok{,}\DecValTok{1}\NormalTok{,}\DataTypeTok{length =} \DecValTok{1000}\NormalTok{)}
\NormalTok{x <-}\StringTok{ }\KeywordTok{sum}\NormalTok{(X)}
\NormalTok{n <-}\StringTok{ }\KeywordTok{length}\NormalTok{(X)}
\NormalTok{a =}\StringTok{ }\FloatTok{0.1}
\NormalTok{b =}\StringTok{ }\FloatTok{1.0}

\CommentTok{# Generate the prior and post function according to conclusions above}
\NormalTok{prior <-}\StringTok{ }\KeywordTok{dgamma}\NormalTok{(th, a, b)}
\NormalTok{post <-}\StringTok{ }\KeywordTok{dgamma}\NormalTok{(th, a }\OperatorTok{+}\StringTok{ }\NormalTok{n, b }\OperatorTok{+}\StringTok{ }\NormalTok{x)}

\CommentTok{# Plotting}
\KeywordTok{ggplot}\NormalTok{()}\OperatorTok{+}\KeywordTok{geom_line}\NormalTok{(}\KeywordTok{aes}\NormalTok{(}\DataTypeTok{x =}\NormalTok{ th, }\DataTypeTok{y =}\NormalTok{ prior, }\DataTypeTok{linetype =} \StringTok{'prior'}\NormalTok{))}\OperatorTok{+}
\StringTok{  }\KeywordTok{geom_line}\NormalTok{(}\KeywordTok{aes}\NormalTok{(}\DataTypeTok{x =}\NormalTok{ th, }\DataTypeTok{y =}\NormalTok{ post, }\DataTypeTok{linetype =} \StringTok{'post'}\NormalTok{))}\OperatorTok{+}
\StringTok{  }\KeywordTok{labs}\NormalTok{(}\DataTypeTok{x =} \KeywordTok{expression}\NormalTok{(theta), }\DataTypeTok{y =} \StringTok{'density'}\NormalTok{, }\DataTypeTok{title =} \StringTok{'Posterior and Prior based on given data'}\NormalTok{) }\OperatorTok{+}\StringTok{ }\KeywordTok{theme_bw}\NormalTok{()}
\end{Highlighting}
\end{Shaded}

\includegraphics{HW2_files/figure-latex/unnamed-chunk-4-1.pdf}

\paragraph{(d) Give a specific example of an application where an
Exponential model would be reasonable. Give an example where an
Exponential model would NOT be appropriate, and explain
why.}\label{d-give-a-specific-example-of-an-application-where-an-exponential-model-would-be-reasonable.-give-an-example-where-an-exponential-model-would-not-be-appropriate-and-explain-why.}

Reasonable: The time that some radioactive matter decays.

Unreasonable: The amount of flying bugs in different hours of day.
Because most bugs are only active during night time, which is not
constant throughout the day.

\paragraph{\texorpdfstring{3. \textbf{\emph{Priors, Posteriors,
Predictive Distributions (Hoff,
3.9)}}.}{3. Priors, Posteriors, Predictive Distributions (Hoff, 3.9).}}\label{priors-posteriors-predictive-distributions-hoff-3.9.}

\paragraph{\texorpdfstring{An unknown quantity \(Y\) has a
Galenshore(\(a, \theta\)) distribution if its density is given
by}{An unknown quantity Y has a Galenshore(a, \textbackslash{}theta) distribution if its density is given by}}\label{an-unknown-quantity-y-has-a-galenshorea-theta-distribution-if-its-density-is-given-by}

\[p(y) = \frac{2}{\Gamma(a)} \; \theta^{2a} y^{2a - 1} e^{-\theta^2 y^2}\]
for \(y>0, \theta >0, a>0.\) Assume for now that \(a\) is known. For
this density, \[E[Y] = \frac{\Gamma(a +1/2)}{\theta \Gamma(a)}\] and
\[E[Y^2] = \frac{a}{\theta^2}.\]

\paragraph{\texorpdfstring{(a) Identify a class of conjugate prior
densities for \(\theta\). Plot a few members of this class of
densities.}{(a) Identify a class of conjugate prior densities for \textbackslash{}theta. Plot a few members of this class of densities.}}\label{a-identify-a-class-of-conjugate-prior-densities-for-theta.-plot-a-few-members-of-this-class-of-densities.}

\begin{Shaded}
\begin{Highlighting}[]
\CommentTok{# Define dgalen function}
\NormalTok{dgalen <-}\StringTok{ }\ControlFlowTok{function}\NormalTok{(th,a,b)\{}
  \KeywordTok{return}\NormalTok{(}\DecValTok{2}\OperatorTok{/}\KeywordTok{gamma}\NormalTok{(a)}\OperatorTok{*}\NormalTok{b}\OperatorTok{^}\NormalTok{(}\DecValTok{2}\OperatorTok{*}\NormalTok{a)}\OperatorTok{*}\NormalTok{th}\OperatorTok{^}\NormalTok{(}\DecValTok{2}\OperatorTok{*}\NormalTok{a}\OperatorTok{-}\DecValTok{1}\NormalTok{)}\OperatorTok{*}\KeywordTok{exp}\NormalTok{(}\OperatorTok{-}\NormalTok{b}\OperatorTok{^}\DecValTok{2}\OperatorTok{*}\NormalTok{th}\OperatorTok{^}\DecValTok{2}\NormalTok{))}
\NormalTok{\}}
\CommentTok{# Creating a set of (a, b) and theta for plotting}
\NormalTok{a1 =}\StringTok{ }\DecValTok{1}
\NormalTok{b1 =}\StringTok{ }\DecValTok{1}
\NormalTok{a2 =}\StringTok{ }\DecValTok{3}
\NormalTok{b2 =}\StringTok{ }\DecValTok{1}
\NormalTok{a3 =}\StringTok{ }\DecValTok{1}
\NormalTok{b3 =}\StringTok{ }\DecValTok{3}
\NormalTok{th =}\StringTok{ }\KeywordTok{seq}\NormalTok{(}\DecValTok{0}\NormalTok{, }\DecValTok{5}\NormalTok{, }\DataTypeTok{length =} \DecValTok{1000}\NormalTok{)}

\CommentTok{# Plotting}
\KeywordTok{ggplot}\NormalTok{()}\OperatorTok{+}\KeywordTok{geom_line}\NormalTok{(}\KeywordTok{aes}\NormalTok{(}\DataTypeTok{x=}\NormalTok{th, }\DataTypeTok{y=}\KeywordTok{dgalen}\NormalTok{(th,a1,b1),}\DataTypeTok{linetype=}\StringTok{'1'}\NormalTok{))}\OperatorTok{+}
\StringTok{  }\KeywordTok{geom_line}\NormalTok{(}\KeywordTok{aes}\NormalTok{(}\DataTypeTok{x=}\NormalTok{th, }\DataTypeTok{y=}\KeywordTok{dgalen}\NormalTok{(th,a2,b2),}\DataTypeTok{linetype=}\StringTok{'2'}\NormalTok{))}\OperatorTok{+}
\StringTok{  }\KeywordTok{geom_line}\NormalTok{(}\KeywordTok{aes}\NormalTok{(}\DataTypeTok{x=}\NormalTok{th, }\DataTypeTok{y=}\KeywordTok{dgalen}\NormalTok{(th,a3,b3),}\DataTypeTok{linetype=}\StringTok{'3'}\NormalTok{))}\OperatorTok{+}
\StringTok{  }\KeywordTok{theme_bw}\NormalTok{()}
\end{Highlighting}
\end{Shaded}

\includegraphics{HW2_files/figure-latex/unnamed-chunk-5-1.pdf}

\paragraph{\texorpdfstring{(b) Let
\(Y_1, \ldots, Y_n \stackrel{iid}{\sim}\) Galenshore(\(a, \theta\)).
Find the posterior distribution of \(\theta \mid y_{1:n}\) using a prior
from your conjugate
class.}{(b) Let Y\_1, \textbackslash{}ldots, Y\_n \textbackslash{}stackrel\{iid\}\{\textbackslash{}sim\} Galenshore(a, \textbackslash{}theta). Find the posterior distribution of \textbackslash{}theta \textbackslash{}mid y\_\{1:n\} using a prior from your conjugate class.}}\label{b-let-y_1-ldots-y_n-stackreliidsim-galenshorea-theta.-find-the-posterior-distribution-of-theta-mid-y_1n-using-a-prior-from-your-conjugate-class.}

Suppose the prior, \(\theta\), is \(\theta \sim\) Galenshore(k, b),

Given that Y has a Galenshore(a, \(\theta\)) distribution for
y1,\ldots{},yn,

then the posterior is:

\[\begin{align}p(\theta|y_{1:n}) & \propto p(\theta)p(y_{1:n}|\theta)\\
& \propto \theta^{2k-1}\exp(-b^2\theta^2 )\prod_n [\theta^{2a}yi^{2a-1}\exp(-\theta^2yi^2)]\\
& \propto \theta^{2k-1}\exp(-b^2\theta^2 )\theta^{2an}\exp(-\theta^2\sum_n yi^2)\\
& \propto \theta^{2k+2an-1}\exp[-\theta^2(b^2+\sum_n yi^2)]\\
& \propto Galenshore(\alpha = k+an,\beta = \sqrt{b^2+\sum_n yi^2})
\end{align}\]

\paragraph{\texorpdfstring{(c) Write down
\[\frac{p(\theta_a \mid y_{1:n})}{p(\theta_b \mid y_{1:n})}\] and
simplify. Identify a sufficient
statistic.}{(c) Write down \textbackslash{}frac\{p(\textbackslash{}theta\_a \textbackslash{}mid y\_\{1:n\})\}\{p(\textbackslash{}theta\_b \textbackslash{}mid y\_\{1:n\})\} and simplify. Identify a sufficient statistic.}}\label{c-write-down-fracptheta_a-mid-y_1nptheta_b-mid-y_1n-and-simplify.-identify-a-sufficient-statistic.}

Given the result from (b),
\(p(\theta|y_{1:n}) \propto Galenshore(\alpha = k+an,\beta = \sqrt{b^2+\sum_n yi^2})\):

\[\frac{p(\theta_a \mid y_{1:n})}{p(\theta_b \mid y_{1:n})} = \frac{\theta_a^{2\alpha-1}exp(-\beta^2\theta_a^2)}{\theta_b^{2\alpha-1}exp(-\beta^2\theta_b^2)} = \bigg( \frac{\theta_a}{\theta_b}\bigg)^{2\alpha-1}exp[-\beta^2(\theta_a^2-\theta_b^2)]\]

Sufficient statistic: \(\sum_n yi^2\)

\paragraph{\texorpdfstring{(d) Determine
\(E[\theta \mid y_{1:n}]\).}{(d) Determine E{[}\textbackslash{}theta \textbackslash{}mid y\_\{1:n\}{]}.}}\label{d-determine-etheta-mid-y_1n.}

Since this is a conjugate family, and since given
\(Y \sim Galenshore(a,\theta)\),
\[E[Y] = \frac{\Gamma(a +1/2)}{\theta \Gamma(a)}\]

\[E[\theta \mid y_{1:n}] = \frac{\Gamma(\alpha +1/2)}{\beta \Gamma(\alpha)}\\
given: \alpha = k+an,\beta = \sqrt{b^2+\sum_n yi^2},\\
where: \theta \sim Galenshore(k, b)\]

\paragraph{\texorpdfstring{(e) Determine the form of the posterior
predictive density
\(p(y_{n+1} \mid y_{1:n})\)}{(e) Determine the form of the posterior predictive density p(y\_\{n+1\} \textbackslash{}mid y\_\{1:n\})}}\label{e-determine-the-form-of-the-posterior-predictive-density-py_n1-mid-y_1n}

\[\begin{align} p(y_{n+1} \mid y_{1:n}) &= \int p(y_{n+1} |\theta) p(\theta|y_{1:n})d\theta\\
&= \int \frac{2}{\Gamma(a)} \; \theta^{2a} y_{n+1}^{2a - 1} e^{-\theta^2 y_{n+1}^2}\frac{2}{\Gamma(\alpha)} \; \beta^{2\alpha} \theta^{2\alpha - 1} e^{-\beta^2 \theta^2}d\theta\\
&= \frac{4}{\Gamma(a)\Gamma{(\alpha)}}y_{n+1}^{2a - 1}\beta^{2\alpha} \int \theta^{2(a+\alpha)-1} e^{-\theta^2(y_{n+1}^2 + \beta^2)}d\theta\\
&= \frac{4}{\Gamma(a)\Gamma{(\alpha)}}y_{n+1}^{2a - 1}\beta^{2\alpha} \int \frac{\Gamma(a+\alpha-\frac{1}{2})}{2}(y_{n+1}^2 + \beta^2)^{-(a+\alpha-\frac{1}{2})}\frac{2}{\Gamma(a+\alpha-\frac{1}{2})}(y_{n+1}^2 + \beta^2)^{a+\alpha-\frac{1}{2}}\theta^{2(a+\alpha-\frac{1}{2})-1} e^{-\theta^2(y_{n+1}^2 + \beta^2)}\theta d\theta\\
&= \frac{2\Gamma(a+\alpha-\frac{1}{2})}{\Gamma(a)\Gamma{(\alpha)}}y_{n+1}^{2a - 1}\beta^{2\alpha}(y_{n+1}^2 + \beta^2)^{-(a+\alpha-\frac{1}{2})}\int Galenshore(\theta|a+\alpha-\frac{1}{2}, \sqrt{y_{n+1}^2 + \beta^2})\theta d\theta\\
&\propto y_{n+1}^{2a - 1}(y_{n+1}^2 + \beta^2)^{-(a+\alpha-\frac{1}{2})}\frac{\Gamma(a+\alpha)}{\sqrt{y_{n+1}^2 + \beta^2}\Gamma(a+\alpha-\frac{1}{2})}\\
&\propto y_{n+1}^{2a - 1}(y_{n+1}^2 + \beta^2)^{-(a+\alpha)}
\end{align}\]

This is the simpliest form I can get, but I cannot recognize the family
of the distribution.


\end{document}
