\documentclass{article}
\usepackage{hyperref}
% Custom definitions
% To use this customization file, insert the line "\input{custom}" in the header of the tex file.

% Formatting




% Packages

 \usepackage{amssymb,latexsym}
\usepackage{amssymb,amsfonts,amsmath,latexsym,amsthm, bm}
%\usepackage[usenames,dvipsnames]{color}
%\usepackage[]{graphicx}
%\usepackage[space]{grffile}
\usepackage{mathrsfs}   % fancy math font
% \usepackage[font=small,skip=0pt]{caption}
%\usepackage[skip=0pt]{caption}
%\usepackage{subcaption}
%\usepackage{verbatim}
%\usepackage{url}
%\usepackage{bm}
\usepackage{dsfont}
\usepackage{multirow}
%\usepackage{extarrows}
%\usepackage{multirow}
%% \usepackage{wrapfig}
%% \usepackage{epstopdf}
%\usepackage{rotating}
%\usepackage{tikz}
%\usetikzlibrary{fit}					% fitting shapes to coordinates
%\usetikzlibrary{backgrounds}	% drawing the background after the foreground


% \usepackage[dvipdfm,colorlinks,citecolor=blue,linkcolor=blue,urlcolor=blue]{hyperref}
%\usepackage[colorlinks,citecolor=blue,linkcolor=blue,urlcolor=blue]{hyperref}
%%\usepackage{hyperref}
%\usepackage[authoryear,round]{natbib}


%  Theorems, etc.

%\theoremstyle{plain}
%\newtheorem{theorem}{Theorem}[section]
%\newtheorem{corollary}[theorem]{Corollary}
%\newtheorem{lemma}[theorem]{Lemma}
%\newtheorem{proposition}[theorem]{Proposition}
%\newtheorem{condition}[theorem]{Condition}
% \newtheorem{conditions}[theorem]{Conditions}

%\theoremstyle{definition}
%\newtheorem{definition}[theorem]{Definition}
%% \newtheorem*{unnumbered-definition}{Definition}
%\newtheorem{example}[theorem]{Example}
%\theoremstyle{remark}
%\newtheorem*{remark}{Remark}
%\numberwithin{equation}{section}




% Document-specific shortcuts
\newcommand{\btheta}{{\bm\theta}}
\newcommand{\bbtheta}{{\pmb{\bm\theta}}}

\newcommand{\commentary}[1]{\ifx\showcommentary\undefined\else \emph{#1}\fi}

\newcommand{\term}[1]{\textit{\textbf{#1}}}

% Math shortcuts

% Probability distributions
\DeclareMathOperator*{\Exp}{Exp}
\DeclareMathOperator*{\TExp}{TExp}
\DeclareMathOperator*{\Bernoulli}{Bernoulli}
\DeclareMathOperator*{\Beta}{Beta}
\DeclareMathOperator*{\Ga}{Gamma}
\DeclareMathOperator*{\TGamma}{TGamma}
\DeclareMathOperator*{\Poisson}{Poisson}
\DeclareMathOperator*{\Binomial}{Binomial}
\DeclareMathOperator*{\NormalGamma}{NormalGamma}
\DeclareMathOperator*{\InvGamma}{InvGamma}
\DeclareMathOperator*{\Cauchy}{Cauchy}
\DeclareMathOperator*{\Uniform}{Uniform}
\DeclareMathOperator*{\Gumbel}{Gumbel}
\DeclareMathOperator*{\Pareto}{Pareto}
\DeclareMathOperator*{\Mono}{Mono}
\DeclareMathOperator*{\Geometric}{Geometric}
\DeclareMathOperator*{\Wishart}{Wishart}

% Math operators
\DeclareMathOperator*{\argmin}{arg\,min}
\DeclareMathOperator*{\argmax}{arg\,max}
\DeclareMathOperator*{\Cov}{Cov}
\DeclareMathOperator*{\diag}{diag}
\DeclareMathOperator*{\median}{median}
\DeclareMathOperator*{\Vol}{Vol}

% Math characters
\newcommand{\R}{\mathbb{R}}
\newcommand{\Z}{\mathbb{Z}}
\newcommand{\E}{\mathbb{E}}
\renewcommand{\Pr}{\mathbb{P}}
\newcommand{\I}{\mathds{1}}
\newcommand{\V}{\mathbb{V}}

\newcommand{\A}{\mathcal{A}}
%\newcommand{\C}{\mathcal{C}}
\newcommand{\D}{\mathcal{D}}
\newcommand{\Hcal}{\mathcal{H}}
\newcommand{\M}{\mathcal{M}}
\newcommand{\N}{\mathcal{N}}
\newcommand{\X}{\mathcal{X}}
\newcommand{\Zcal}{\mathcal{Z}}
\renewcommand{\P}{\mathcal{P}}

\newcommand{\T}{\mathtt{T}}
\renewcommand{\emptyset}{\varnothing}


% Miscellaneous commands
\newcommand{\iid}{\stackrel{\mathrm{iid}}{\sim}}
\newcommand{\matrixsmall}[1]{\bigl(\begin{smallmatrix}#1\end{smallmatrix} \bigr)}

\newcommand{\items}[1]{\begin{itemize} #1 \end{itemize}}

\newcommand{\todo}[1]{\emph{\textcolor{red}{(#1)}}}

\newcommand{\branch}[4]{
\left\{
	\begin{array}{ll}
		#1  & \mbox{if } #2 \\
		#3 & \mbox{if } #4
	\end{array}
\right.
}

% approximately proportional to
\def\app#1#2{%
  \mathrel{%
    \setbox0=\hbox{$#1\sim$}%
    \setbox2=\hbox{%
      \rlap{\hbox{$#1\propto$}}%
      \lower1.3\ht0\box0%
    }%
    \raise0.25\ht2\box2%
  }%
}
\def\approxprop{\mathpalette\app\relax}

% \newcommand{\approptoinn}[2]{\mathrel{\vcenter{
  % \offinterlineskip\halign{\hfil$##$\cr
    % #1\propto\cr\noalign{\kern2pt}#1\sim\cr\noalign{\kern-2pt}}}}}

% \newcommand{\approxpropto}{\mathpalette\approptoinn\relax}






\begin{document}
\title{Homework 3}
\author{STA-360/602, Spring 2018}
\date{Due at 10:00 AM on Monday, 28 January 2019}
\maketitle

\textbf{General instructions for homeworks}: Please follow the uploading file instructions according to the syllabus. You will give the commands to answer each question in its own code block, which will also produce plots that will be automatically embedded in the output file. Each answer must be supported by written statements as well as any code used. Your code must be completely reproducible and must compile. Syllabus: (https://github.com/resteorts/modern-bayes/blob/master/syllabus/syllabus-sta602-spring19.pdf)

\textbf{Advice}: Start early on the homeworks and it is advised that you not wait until the day of. While the professor and the TA's check emails, they will be answered in the order they are received and last minute help will not be given unless we happen to be free.  

\textbf{Commenting code}
Code should be commented. See the Google style guide for questions regarding commenting or how to write 
code \url{https://google.github.io/styleguide/Rguide.xml}. No late homework's will be accepted.

%This homework has a maximum possible score of 100;
%90 points on this are for correctness of your answers (as marked), and 10 for
%over-all clarity and thoroughness.

Please note that this homework has a set of ungraded homework exercises to
help prepare you for exam 1 (exercises, 2--4). You can find the additional exercises at \url{https://github.com/resteorts/modern-bayes/tree/master/exercises.}


\begin{enumerate}
\item {\em Lab component} 
  (90 points total) Please refer to module 2 and lab 3 and complete tasks 3---5. 
  \begin{enumerate}
  \item (40) Task 3
  \item (40) Task 4
  \item (10) Task 5
  \end{enumerate}
  
  \newpage
  
Practice problems for exam 1 (not to be graded).   
  
\item (15 points total) {\em The Uniform-Pareto}

Suppose $a < x < b.$ Consider the notation
$I_{(a,b)}(x),$ where $I$ denotes the indicator function. We define $I_{(a,b)}(x)$ to be the following:
$$
I_{(a,b)}(x)=
\begin{cases} 
1 & \text{if $a < x < b$,}
\\
0 &\text{otherwise.}
\end{cases}
$$

Let 
\begin{align*}
X|\theta &\sim \text{Uniform}(0,\theta)\\
\theta &\sim \text{Pareto}(\alpha,\beta),
\end{align*}
where $p(\theta) = \dfrac{\alpha \beta^\alpha}{\theta^{\alpha +1}}I_{(\beta,\infty)}(\theta).$
Write out the likelihood $p(x\mid \theta).$ Then calculate the posterior distribution of $\theta|x.$  
  
  
\item (15)  points total) {\em The Bayes estimator or Bayes procedure}
\begin{enumerate}
\item (5 pts) Find the Bayes estimator (or Bayes procedure) when the loss function is  $L(\theta, \delta(x))~=~c~(\theta-\delta(x))^2,$ where $c$ is a constant. 
\item (10 pts) Derive the Bayes estimator (or Bayes procedure) when $L(\theta, \delta(x)) = w(\theta) (g(\theta)-\delta(x))^2.$ Do so without writing any integrals. Note that you can write $\rho(\pi,\delta(x)) =  E[L(\theta,\delta(x))|X].$ 
\end{enumerate}

\item  (10 points total) {\em Basic decision theory}

Consider the decision problem in which $\Theta = \{ \theta_1, \theta_2\}, 
A = \{a_1, a_2, a_3, a_4, a_5\}$, and the loss function is given as follows: 
$$L(\theta_1, a_1) = 0,\quad L(\theta_1, a_2) = 3,\quad L(\theta_1, a_3) = 1,\quad L(\theta_1, a_4) = 3,\quad
L(\theta_1, a_5) = 4;$$ 
$$L(\theta_2, a_1) = 4,\quad L(\theta_2, a_2) = 6,\quad L(\theta_2, a_3) = 0,\quad L(\theta_2, a_4) = 0,\quad
L(\theta_2, a_5) = 1.$$
Consider the prior $\pi$ under which $\pi(\theta_1) = 4/5$ and $\pi(\theta_2) = 1/5.$ Find the Bayes action(s) or rule(s) under this prior. 


    \end{enumerate}

\end{document}
